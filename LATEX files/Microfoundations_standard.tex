\title{The Economic Logic of Online Community Management: an Empirical Study}
\author{Alberto Cottica}

\documentclass{article}
\usepackage{graphicx}
\usepackage{subcaption}
\usepackage{mathtools}
\usepackage{hyperref}
\graphicspath{ {./Paper2Images/} }

\setlength{\parindent}{0em}
\setlength{\parskip}{1em}

\begin{document}

\maketitle

	\begin{abstract}
		Online communities are pervasive in many contexts, such as business, politics, expertise sharing and education. We model them as networks of interactions; each network's topology emerges from the pairwise interactions of its members. Almost all online communities are initiated and run by organisations; the latter typically employ professionals, online community managers, to encourage participation and resolve conflicts. We conjecture the role of online community managers is to nudge the (emergent) network's topology towards a state that servers the organisation purposes. Online community managers do this by interacting with users of the community. 
		Though the practice of employing online community managers is pervasive, there is little evidence that their work makes a measurable difference in user behaviour. We devise a test, based on panel data econometrics, to ascertain whether interaction with community managers is likely to make users more active; and implement it on a small online community. We find that, indeed, interaction with online community managers has a positive, strongly significant effect on user activity. We then estimate its marginal effect, and find it is relatively large, but rapidly decreasing in the number of interactions per period.
	\end{abstract}

	\section{Introduction}

Online communities are used to aggregate and process information dispersed across many individuals. Pioneered in the 1980s, they have become more widespread with mass adoption of the Internet, and are now used across many different contexts in business \cite{mcwilliam2012building, tapscott2008wikinomics}, politics and public decision making \cite{rheingold1993virtual, noveck2009wiki, cottica2010wikicrazia}, expertise sharing \cite{rheingold1993virtual, zhang2007expertise, shirky2008here}, and education \cite{milligan2013patterns}. At the same time as they spread across domains, they did so geographically: for example, they have attracted large numbers of users and large venture capital investments in China \cite{zhou2011social}. Most online communities lack a central command structure; despite this, many display remarkably coherent behaviour, and have proven effective at large tasks like writing the largest encyclopedia in human history (Wikipedia), providing an always-on free helpline for software engineering problems (StackOverflow), or building, and continuously updating, a detailed map of planet Earth (OpenStreetMap) \cite{shirky2008here}. 

Organizations running online communities typically employ community managers, tasked with encouraging participation and resolving conflict: this practice is almost as old as online communities themselves and predates the Internet \cite{rheingold1993virtual}, although it has become much more widespread as Internet access became a mass phenomenon. Though most participants to online communities are unpaid and answer to no one, a small number of them (only one or two in the smaller communities, many more in the larger ones) report to a central command, and carry out its directives. Following the convention of practitioners themselves, we shall henceforth call such directives \emph{policies}. 

Putting in place policies for online communities is costly. Professional community managers need to be recruited, trained and paid; software tools to monitor communities and make their work possible need to be developed and maintained. This raises the question of what benefits organisations running online communities expect from policies; and why they choose certain policies, and not others. 

A full investigation of this matter is outside the scope of this paper; however, in what follows we outline and briefly discuss the set of assumptions that underpin our investigation. 

\begin{enumerate}
\item In line with the network science approach to online communities, we model online communities as social networks of interactions across participants. 
\item We assume that organisations can be modelled as economic agents maximising some objective function. The target variable being maximised can be profit (for online communities run by commercial companies); or welfare (for online communities run by governments or other nonprofit entities); or some combination of the two. 
\item We assume that the topology of the interaction network characteristic of online communities affects their ability to contribute to the maximisation of the target variable. 
\item We assume that such organisations choose their policies as follows: 
\begin{itemize} 
	\item Solve their maximisation problem over network topology. This yields a vector of desired network characteristics, where "desired" means that those characteristics define a maximum of the objective function. These solutions will be statements with the form "In order to best meet our ultimate [profit or welfare] goals, the interaction network in our online community should be in state $\Theta_D$, where $\Theta$ is a vector of topology-related parameters".
	\item Derive a course of action that community managers could take to change the network away from its present state $\Theta_0$ to the desired state $\Theta_D$.
	\item Encode such course of action in a set of simple instructions for community managers to execute. They call them policies;  computer scientists might think of such instructions as algorithms; economists call them mechanisms. 
\end{itemize}
\end{enumerate}

All this implies that the organisation running the online community has tools at its disposal to (attempt to) reach such preferred states of the interaction network. But rarely, if ever, is it explicitly argued that, in fact, it does. Network topology results from the uncoordinated behaviour of all users of the online community. Most of them have joined the community by their own free will, can leave at any time, and can in no way be forced to comply with the organisation's directives. This makes topology an emergent property of the pattern of interaction. For recommendations about "better" states to be meaningful, we need to verify that organisations have the tools to achieve them. 

This paper attempts to bridge that gap. To do so, it does not attempt to model the whole chain of decision starting from the maximisation problem. Rather, it focuses on those participants in an online community that \textit{are} answerable to the organisation in charge of it: its online community managers. We conjecture the following:

\begin{enumerate}
 	\item Organisations can formulate policies and instruct online community managers to execute them.
	 \item Online community managers execute by communicating with users.
	 \item Communication with online community managers nudges users towards taking the course of action desired by the organisation. 
 \end{enumerate}
 
The first two items in this list are assumed to fit into the mechanism design framework. This entails assuming that the organisation in charge of the online community knows both the desired network state $\Theta_D$ and the set of behaviours that, were they adopted by the online community's users, would result in it achieving $\Theta_D$. We focus on the final item, which is equivalent to assuming that user behaviour is responsive to communication with online community managers. This assumption is implicit both in the literature and in managerial practice but, to the best of our knowledge, has not yet been tested.

We consider a policy aimed at user activation: it consists of commenting a unit of content authored by a user. The comment could read something like the one below:

\emph{``Hello, Alice! That was a very interesting point. It definitely resonates with my own experience in the field. In our community, the people who are most involved in the matter are Bob [link] and Charlie [link]. You might be interested in this post [link] by Bob, where he relates his own experience: if you leave him a comment, I am sure an interesting conversation will ensue.''}

It is successful if the user becomes active in the current period.  

Many online communities take care to do this systematically with new users. As a new participant becomes active (for example by posting her first post, or commenting somebody else's post for the first time), community managers are instructed to leave her a comment that contains (a) friendly, positive feedback and (b) suggestions to engage with other, existing participants that she might have interests in common with,\footnote{Katherina Fake, founder of Flickr, a popular website to share photographs, is reported to have deployed the company's employees as the website's first users and the initial core of its community. According to her "We learned you have to greet the first ten thousand users personally". \cite{shirky2008here}} perhaps one of the simplest and most common in online community management \cite{rheingold1993virtual, shirky2008here}. 

In the remainder of the paper, we consider the issue of whether user behaviour does, indeed, respond to activation policies. Organizations running online communities invest considerable resources in them, and user responsiveness is an important parameter in making the decision of how much to spend in this activity as opposed to other, competing ones.  

Section 2 briefly examines the literature that we mostly draw upon. Section 3 presents some data from a real-world online community called Edgeryders; then proceeds to specify a model of that community's user behaviour in the presence of activation policies. Section 4 presents the estimation's results. Section 5 discusses them.
	\section{Literature survey}

We are interested in the responsiveness of the behaviour of participants in online communities when policies, aiming to change the topology of the interaction network to a desired state, are enacted by the organisations in charge of the online community itself. This topic finds itself at the intersection of two different strands of academic literature.

The first strand is mainly interested in the determinants of the behaviour of individuals participating in online communities. Many scholars, moving from a psychology or business studies background, take the view that individual behaviour in online community both responds to social norms and contributes to shape them. Several attempts have been made to conceptualize this view into testable models (\cite{armstrong2000online}, \cite{dholakia2004social}, \cite{zhou2011social}). Such attempts use survey data to estimate structural equation models. The models' parameters quantify the influence of both individual incentives (such as obtaining valued information) and social incentives (such as group norms) in participating in online communities. 

The second strand moves from a social network science background, and looks rather at the interplay between network topology and user behaviour. The best-known example is probably Ronald Burt's theory of structural holes (\cite{burt2005brokerage}), which generalises to any community, both online and offline ones. There exist several studies dedicated to specific online communities, such as Slashdot (\cite{toral2009empirical}), Java developers forums (\cite{zhang2007expertise}) and Linux ports developer forums (\cite{ganley2009ties}). Some authors have attempted to merge these two strands of literature, treating network topology characteristics as variables to incorporate into their structural equation models (\cite{toral2009empirical}, \cite{ganley2009ties}).

Authors from both strands agree that some topological characteristics are more conducive than others to the organisation's goals. For example, Burt (\cite{burt2005brokerage}) suggests that densely connected clusters of individuals are useful to better focus on goals and targets, whereas less dense networks with some individuals bridging across clusters are more conducive to innovation; Ganley and Lampe (\cite{ganley2009ties}) propose that densely connected clusters of "power users" lead to social tensions and a loss of the egalitarian spirit that makes many online communities attractive; Kim and collaborators (\cite{kim2015group}) show that, in some circumstances, a tendency towards communication reciprocity ("intimacy") leads to membership loss and, potentially, network breakdown. Dholakia and collaborators are perhaps clearest in indicating that their findings are relevant to designing and enacting policies:

\begin{quotation}
Understanding the antecedents of social influence is important since it is likely to provide significant managerial guidance regarding how to make virtual communities useful and influential for their participants. (\cite{dholakia2004social}, p. 242) 
\end{quotation}

Most authors do not explicitly envision pathways leading from the formulation of a policy to attaining the desired change in network topology. Some do point to the relevance of online community managers, explicitly (\cite{dholakia2004social}) or implicitly (\cite{toral2009empirical}), by referring to cases in which such professionals are prominent (\cite{rheingold1993virtual}); but we have been unable to find an explicit discussion of how few online community manager can wilfully alter the behaviour of the many participants to the same community. Departing from this practice, in a previous paper (\cite{cottica2015online}), we have proposed such a pathway for the onboarding policy: online community managers communicate one-to-one with new members of the online community and suggest they initiate communication with existing members, indicating those existing members that seem to match the newcomer's interests best. This paper provides an empirical test that such behaviour does indeed prompt users to interact with each other, underpinning the usefulness and profitability of engaging in online community management activities.  

The present paper is also related to the growing mass of literature on the treatment of time in network analysis. Early work in networks, both in mathematics (for example \cite{erdds1959random}) and in sociology (for example \cite{moreno1937sociometry}), focused on the topology of static networks. The beginning of the 21st century marked a surge of interest in evolving networks (for example \cite{barabasi1999emergence} and related work; see \cite{dorogovtsev2002evolution} for a survey). Scholars sharing this interest investigate the growth paths that produce certain notable topologies often encountered in real-world networks. Later still, a literature on dynamic networks developed, interested in dynamical processes that happen in networks, like information diffusion (for example \cite{eckmann2004entropy} or epidemics (for example \cite{rocha2011simulated}). Many of these studies direct their attention towards the statistical properties of the sequences of event that make up spreading dynamics. Human communication dynamics, they find, turns out to be "bursty", with the timings between communication events deviating significantly from uniform or Poissonian statistics  (see \cite{holme2012temporal} for a survey). 

Our focus does not lie in the dynamics of communication events in online communities. Rather, we wish to establish whether  community managers can really influence the behaviour of participants in the online community. The pattern of relationships across the community is likely to have some influence on users' willingness to initiate communication. Since such patterns varies across time, we adopt the stance of representing temporal data as a sequence of static graphs, tracing the evolution of the interaction network. This allows us to treat the topology of the network as a control variable, whose effect on the users likelihood to initiate communication must be kept separate from that of the action of community managers. It is a common enough technique, deemed appropriate when, as is our case, topology is more central to the analysis than time sequence (\cite{rosvall2010mapping}).

	\input{methods.tex}
 	\section{Results}

\subsection{Hypotheses}

We wish to test the following hypotheses.

\newtheorem{policyWorks}{Hypothesis}

\begin{policyWorks}
	The number of comments a user receives from community managers has no influence on her probability to be active.
	\label{hypothesis:policyWorks}
\end{policyWorks}

The organisations running online communities wish for their users to do certain things, most of which imply being active in the community itself: writing posts and comments. They cannot order them to do so, since users are not on the payroll and remain unanswerable to those organisations. Online community managers are then tasked to prompt users into action without using either monetary incentives or command power. That leaves interaction as the main tool online community managers have at their disposal. Rejecting Hypothesis \ref{hypothesis:policyWorks} would imply that users do, in fact, respond to cues from online community managers.

We expect Hypothesis \ref{policyWorks} to be falsified by data. Moreover, we expect that the coefficient on the number of comments received by the user from community managers, besides being statistically different from zero, is positive. Denote said coefficient by  $\beta_{cmrec}$, and the related variable by $x_{cmrec}$. Check that: 

\begin{equation}
	\frac{\partial Pr(A=1|x)}{\partial x_{cmrec}} > 0 \Rightarrow \frac{\partial LOR}{\partial x_{cmrec}} > 0
	\label{marginalProbActive}
\end{equation}

where $LOR$ denotes the logarithm of the log-odds ratio as per equation \ref{logOddsErrors}. Since $x_{cmrec}$ enters equation \ref{eq:logOdds} linearly, equation \ref{marginalProbActive} implies that:

\begin{equation}
	\frac{\partial LOR}{\partial x_{cmrec}} > 0 \Rightarrow \beta_{cmrec} > 0 
\end {equation}

\begin{policyWorks}
	Receiving comments from community managers has the same effect on the probability to be active than receiving comments from users that are not community managers. 
	\label{policyWorksBetter}
\end{policyWorks}

Online community managers are professionals. They are likely to have better communication skills than the average user, and they certainly have stronger incentives to craft their interaction modes so as to drive users to being more active in the online community. We therefore expect, on average, the effect of incoming communication from online community manager to have a larger positive effect on the probability of becoming active than that of incoming communication from other users. Therefore, we expect Hypothesis \ref{policyWorksBetter} to be falsified by the data. 

% maybe add a third hypothesis: that the added probability we "buy" by deploying community managers instead of ordinary users is large. Alternative: put this in the results/discussion.

\subsection{Regression}

Table \ref{tab:xtlogit} summarizes the estimation's results. 


\def\onepc{$^{\ast\ast}$} \def\fivepc{$^{\ast}$}
\def\tenpc{$^{\dag}$}
\def\legend{\multicolumn{4}{l}{\footnotesize{Significance levels
:\hspace{1em} $\dag$ : 10\% \hspace{1em}
$\ast$ : 5\% \hspace{1em} $\ast\ast$ : 1\% \normalsize}}}
\begin{table}[htbp]\centering
\begin{tabular}{l r @{} l c }\hline\hline 
\multicolumn{1}{c}
{\textbf{Variable}}
 & \multicolumn{2}{c}{\textbf{Coefficient}}  & \textbf{(Std. Err.)} \\ \hline
number of comments received from community managers  &  1.250&\onepc  & (0.053)\\
number of comments received from other users  &  0.215&\onepc  & (0.037)\\
number of comments and posts written by ego (lagged)  &  0.126&\tenpc  & (0.066)\\
weeks since creating the account  &  -0.011&\tenpc  & (0.006)\\
total number of posts and comments written by users excluding ego  &  0.037&\onepc  & (0.004)\\
total number of posts and comments written by community managers  &  0.001&  & (0.006)\\
user out-degree (lagged) &  0.028&\fivepc  & (0.013)\\
user in-degree (lagged)  &  -0.003&  & (0.013)\\
user betweenness centrality  &  -107.870&\onepc  & (20.778)\\
user pagerank (lagged) &  -5.592&  & (14.610)\\
user clustering coefficient  &  -0.654&\onepc  & (0.195)\\
network average distance  &  -4.156&\fivepc  & (2.007)\\
network average betweenness centrality at $t-1$  &  -9.260&  & (209.212)\\
network Louvain modularity  &  8.379&\onepc  & (2.520)\\
number of nodes  &  -0.001&  & (0.004)\\
number of edges  &  0.000&  & (0.001)\\
\hline
\end{tabular}
 \caption{Estimation results
\label{tab:xtlogit}}
\end{table}


The coefficients on the first two variables are positive and highly significant ($p < 0.001$). This supports the conventional wisdom that users of online communities tend to engage with each other: when made the object of comments, they are more likely to become active than when they are not. By implication, our Hypothesis \ref{policyWorks} cannot be rejected. 

To test Hypothesis \ref{policyWorksBetter}, we start by noting that the coefficient on the number of comments received from community managers is larger than that on the number of comments received from other (non-community managers) users. We next run a Wald test on the null hypothesis that the coefficients on our first two variables are identical. The null is strongly rejected ($p < 0.0001$). This, in turn, means we have no support for Hypothesis \ref{policyWorksBetter}. 

Ideally, we would like that to support the alternative hypothesis that the difference between the coefficient on the number of comments received by the user from community managers and the coefficient on the number of comments received by other (non-community managers) users be positive. Wald tests, being one-tailed, cannot do that. However, we can deduce the non-negativity of such difference from the signs and values of the estimated coefficients and our functional form in equation \ref{logOddsErrors}. Formally, denote the  coefficient on the number of comments received by other (non-community managers) users by $\beta_{urec}$, and the related variable by $x_{urec}$. Since the transformation from probability to odds is monotonic, we have:

\begin{equation}
	\frac{\partial Pr(A=1|x)}{\partial x_{cmrec}} > \frac{\partial Pr(A=1|x)}{\partial x_{urec}}  \Rightarrow \frac{\partial LOR}{\partial x_{cmrec}} > \frac{\partial LOR}{\partial x_{urec}} 
\end{equation}

Applying again equation \ref{logOddsErrors}, and remembering that both $x_{cmrec}$ and $x_{urec}$ enter it linearly, we have:

\begin{equation}
	\frac{\partial LOR}{\partial x_{cmrec}} > \frac{\partial LOR}{\partial x_{urec}}\Rightarrow \beta_{cmrec} > \beta_{urec} \Rightarrow \beta_{cmrec} - \beta_{urec} > 0
	\label{eq:condition4WaldTest}
\end{equation}

So, the difference $\beta_{cmrec} - \beta_{urec}$ cannot be negative. Since we have strongly rejected that it be zero, it follows that it must be strictly positive. 

\subsection{Marginal effects}

Regression analysis shows that the activity of community managers does indeed have a positive impact on the probability that the users they engage will become active. This, however, does not tell us how large that impact is. Since online community management is costly, this is likely to be a question of some relevance to an organisation trying to make a decision to invest in it. 
Coefficient estimates are a poor indicator of marginal effects, because the logit model we employ here is not linear. The relationship between the linear prediction (as defined by the sum of each coefficient multiplied by the regressor it refers to, and assuming the fixed effects are zero) and the probability that the dependent variable is equal to 1 follows a logistic curve (figure \ref{fig:probActivityLinPredict}). An increase in the number of comments received by community managers has a small effect for small or large values of $x\beta$. When $x\beta$ is between -5 and 5, however, the probability that the user becomes active increases quickly as community managers engage more with the user.

\begin{figure}
	\includegraphics[width=.8 \textwidth]{PrAct_linPred}
	\caption{Predicted probability of Edgeryders users to become active as the linear prediction $x\beta$ based on number of comments received by community managers and other users increases.}
\label{fig:probActivityLinPredict}
\end{figure}

Table \ref{table:dydx} shows a point estimate the marginal effect of comments from the two sources (online community managers vs. other users) on the probability that a user will become active. Neither is significant. This turns out to be an artefact of the computation: the size of the marginal effect is computed fixing the value of regressors at their mean. The means of the variables in question are low. In the average period, the average Edgeryders user received 0.08 comments from online community managers and 0.09 comments from other users. This is consistent with what we know about patterns of human communication, which is sparse and bursty (\cite{holme2012temporal}). 

{
\def\onepc{$^{\ast\ast}$} \def\fivepc{$^{\ast}$}
\def\tenpc{$^{\dag}$}
\begin{table}[htbp]\centering
 \begin{tabular}{l c c c}\hline\hline 
\multicolumn{1}{c}
{\textbf{Variable}} & {\textbf{$dy/dx$}}  & \textbf{$Std. Err.$} & \textbf{$P > \lvert z \rvert$} \\ \hline
number of comments received from community managers  &  .0000959  & .0004127 & 0.816 \\
number of comments received from other users  &  .0000165  & .000071 & 0.816\\
\hline
\end{tabular}
\caption{Marginal effects of the number of comments received by a user (both from community managers and from other users) on the probability of that user to become active. The estimates are computed under the assumption that regressors be fixed at their means. 
\label{table:dydx}}
\end{table}
}
A more intuitive approach is to estimate the elasticities of the probability of becoming active with respect to the number of comments received from each source. These are shown in table \ref{tab:elasticities}. They are both highly significant. In absolute terms, they are both small, but quite different. Receiving an extra comment by a community manager increases the probability that a user will become active by 10\%; but receiving one from another user will increase it by less than 2\%.

{
\def\onepc{$^{\ast\ast}$} \def\fivepc{$^{\ast}$}
\def\tenpc{$^{\dag}$}
\def\legend{\multicolumn{4}{l}{\footnotesize{Significance levels
:\hspace{1em} $\dag$ : 10\% \hspace{1em}
$\ast$ : 5\% \hspace{1em} $\ast\ast$ : 1\% \normalsize}}}
\begin{table}[htbp]\centering
 \begin{tabular}{l c c c}\hline\hline 
\multicolumn{1}{c}
{\textbf{Variable}} & {\textbf{$ey/ex$}}  & \textbf{$Std. Err.$} & \textbf{$P > \lvert z \rvert$} \\ \hline
number of comments received from community managers  &  .1024245\onepc  & .0043113 & 0.000 \\
number of comments received from other users  &  .01882\onepc  & .0032799 & 0.000\\
\hline
\end{tabular}\caption{Elasticities of probability that a user becomes active with respect to the number of comments received from community managers and from other users. The estimates are computed under the assumption that regressors be fixed at their means.
\label{tab:elasticities}}
\end{table}
}

Significant elasticities do not, per se, guarantee that the action of community managers will produce a large marginal effect in the online communities, unless the extra action (the differential increase in the regressor) is applied in a region of the distribution where the probability of the user being active is already reasonably high. We therefore turn to computing the marginal effect of the number of comments received from moderators on the probability that a user becomes active. Before we can proceed, however, we need an adjustment. In the model estimated so far, the mean of the predicted probability of a positive outcome (defined as the user
being active in the period) does not coincide with the proportion of users actually active across the dataset (table \ref{tab:flaw}). 

\begin{table}[htbp]\centering
 \begin{tabular}{l c c c}\hline\hline 
\multicolumn{1}{c}
{\textbf{Variable}} & {\textbf{Obs}}  & {\textbf{Mean}} & {\textbf{Std. Dev.}} \\ \hline
$prob$  &    84262   & .0023325  &  .0410453 \\
$active$  &  84262 &   .0181577  &  .1335221 \\
\hline
\end{tabular}\caption{Descriptive statistics for the probability of users to become active as predicted by the model ($prob$) and users actually being active ($active$).
\label{tab:flaw}}
\end{table}

This is caused by fact that the mean of the fixed effects $c_i$ in nonzero. To correct this, we need to add to the model a constant $\alpha$ such that: 

\begin{equation}
	 \frac{e^{mean(X_{i,t}\beta) + \alpha}}{1 + e^{mean(X_{i,t}\beta) + \alpha}} = \frac{\sum_{i = 1}^N \sum_{t = 1}^T active_{i,t}}{NT}
	 \label{eq:equationAlpha}
\end{equation}

In \ref{eq:equationAlpha}, $active_{1,t}$ takes value 1 if user $i$ is active at period $t$, and 0 otherwise; the $x_{i,t}\beta$ are the ones already estimated. The corrected model's linear predictions are unbiased, allowing us to compute marginal effects correctly. The right-hand side of equation \ref{eq:equationAlpha} is identical to the mean of the variable $active$ in table \ref{tab:flaw}. Replacing the appropriate values from table \ref{tab:flaw} yields:

\begin{equation}
	\frac{e^{-9.52 + \alpha}}{1 + e^{-9.52 + \alpha}} = 0.18
\end{equation}

\begin{equation}
	\alpha \simeq 8
	\label{eq:valueAlpha}
\end{equation}

We can now replace \ref{eq:valueAlpha} in to \ref{eq:equationAlpha} and proceed to estimate its marginal effects. Start by noting that, differentiating the right-hand side of \ref{eq:equationAlpha} with respect to $X$ yields:

\begin{equation}
	\frac{\partial Pr(A=1|X)}{\partial X} = \beta \frac{e^{X \beta + \alpha}}{(1 + e^{X \beta + \alpha})^2} 
\end{equation}

Replace the point estimates for $\beta_{cmrec}$, $\beta_{urec}$ and $\alpha$ to yield marginal effects of receiving one additional comments from, respectively, community managers and other users on the probability of being active in the period: 

\begin{equation}
	\frac{\partial Pr(A=1|X)}{\partial x_{cmrec}} = \beta_{cmrec} \frac{e^{ \beta_{cmrec} + \alpha}}{(1 + e^{x_{cmrec} \beta_{cmrec} + \alpha})^2} 
	\label{eq:marginal_x_cmrec}
\end{equation}

\begin{equation}
	\frac{\partial Pr(A=1|X)}{\partial x_{urec}} = \beta_{urec} \frac{e^{ \beta_{urec} + \alpha}}{(1 + e^{x_{urec} \beta_{urec} + \alpha})^2} 
	\label{eq:marginal_x_urec}
\end{equation}

Equations \ref{eq:marginal_x_cmrec} and \ref{eq:marginal_x_urec} allow us to "drill down" into the information contained in Table \ref{table:dydx} and study the marginal effects of receiving one additional comment after the same user has already received zero, one or more comments. To do so, we compute, for each observation and for both $x_{cmrec}$ and $x_{urec}$, the estimate of the marginal effect on the probability of the user being active. These are obtained plugging the observed regressor values, the computed value of $\alpha$ and the point estimates of the coefficient values into respectively, equations \ref{eq:marginal_x_cmrec} and \ref{eq:marginal_x_cmrec}. 

Tabulating 



\begin{table}[htbp]\centering
	\label{tab:marginalProbability_wr_x_cmrec}
	\begin{tabular}{c c | c  c  c  c | c}
 & & \multicolumn{5}{c}{\textbf{Predicted marginal probability of user being active}} \\
N. comments& & 0-0.1&0.1-0.2&0.2-0.3&over 0.3&Total \\
\hline
& prob. & .06 & .15 & .26 & .31 & .23 \\ 
0 & SE & .02 & .03 & .03 & .00 & .07 \\
& N & 5,748&15,303&41,028&19,566&81,645 \\
\hline
 & prob. & .04 & .14 & .25 & .31 & .15 \\
1 & SE & .03 & .03 & .03 & .00 & .10 \\
&N&438&308&362&111&1,219 \\
\hline
& prob. & .03 & .14 & .25 & .31 & .07 \\
2 & SE & .03 & . 03 & .03 & .00 & .08 \\
& N&340&91&32&6&469 \\
\hline
& prob. & .02 & .14 & . 25 & .31 & .04 \\
3 & SE & .02 & .03 & .00 & .07 \\
&N&223&8&13&2&246 \\
\hline
& prob. & .01 & .12 & .25 & .31 & .04 \\
4 & SE & .02 & .01 & .03 & .00 & .07 \\
& N & 135 & 4 & 5 & 2 & 146 \\
\hline
& prob. & .01 & .15 & .25 & .31 & .03 \\
5 to 10 & SE & .02 & .03 &.03 & .00 & .08 \\
& N&329&13&22&8&372 \\
\hline
& prob & .00 & .16 & .26 & .30 & .02 \\
11 and above & SE & .01 & .03 & .04 & 0 & .05 \\
& N &154&6&4&1&165 \\
\hline
& prob. & .05 & .16 & .26 & .31 &.23 \\
Total & SE & .03 & .03 & .03 & .00 & .08\\
& N &7,367&15,733&41,466&19,696&84,262 \\
\hline
	\end{tabular}
	\caption{Distribution of the observations according to the marginal effect of the user being active in the period with respect to receiving an extra comment from a community manager, and of the number of comments received in the same period.}
\end{table}

 	\section{Discussion}

As we have seen, the existence of professional online community managers is predicated on their work solving some optimisation problem for organisations running the online community themselves. Their ability to do so rests ultimately on the influence they can exert on the other participants in the online community. Given the pervasiveness of online community management as a profession, it is perhaps surprising that this influence is generally simply assumed: we have been unable to find explicit empirical tests that this assumption is correct. In this paper, we devise and implement an econometric test for it; we find that interaction with online community managers is associated with a measurably higher probability of users in the online community becoming active. 

We also find that interaction with other (non-community managers users) also increases a user's probability of becoming active; but that this increase is significantly smaller than that associated with the interaction with online community managers. We interpret this as the signature of of the latter's professional skills. 

Our data lend some support to those authors who claim that the topology of the interaction network in online communities influences user behaviour. The probability of a user becoming active depends on some ego network variables (clustering coefficient and betweenness centrality, negatively for both variables) and one global network variable (Louvain modularity, positively) in statistically significant ways. Furthermore, incoming communication is effective in prompting additional (outgoing) communication. This is consistent with the "bursty" nature of human communication described in section \ref{Literature survey}.

In a previous paper (\cite{cottica2015online}), we built a simulation model around the assumption that online community managers could influence the behaviour of other users. That paper makes use of a parameter representing the effectiveness of the community management policy, represented as the increase in probability that users would become active conditional to receiving a communication from one of the community managers. That paper makes no attempt to provide a plausible value for that probability, letting it vary between 0 and 1. With the present work, we are in a position to attempt a quantification. 
 
	\bibliographystyle{plain}
	\bibliography{Microfoundations_standard}

\end{document}